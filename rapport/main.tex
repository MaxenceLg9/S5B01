\documentclass[a4paper,11pt]{report}

%%%%%%%%%% PACKAGES ESSENTIELS %%%%%%%%%%
\usepackage[utf8]{inputenc}   
\usepackage[T1]{fontenc}      
\usepackage[french]{babel}    
\usepackage{graphicx}         
\usepackage{lmodern}          
\usepackage{float}
\usepackage{xcolor}           
\usepackage{geometry}         
\usepackage{caption}
\usepackage{booktabs}
\usepackage{hyperref}
\usepackage{tabularx}
\usepackage{enumitem}
\usepackage{colortbl}        

%%%%%%%%%% PARAMÈTRES DE PAGE %%%%%%%%%%
\geometry{top=1.5cm,bottom=1.5cm,left=2cm,right=2cm}
\setlist{nosep,left=0pt}
\setlength{\parskip}{0.4em}
\setlength{\parindent}{1em}

%%%%%%%%%% ESPACEMENT DES TITRES %%%%%%%%%%
\usepackage{titlesec}
\titlespacing*{\chapter}{0pt}{-5pt}{5pt}
\titlespacing*{\section}{0pt}{6pt}{3pt}
\titlespacing*{\subsection}{0pt}{4pt}{2pt}

%%%%%%%%%% COULEURS & LIENS %%%%%%%%%%
\definecolor{headerblue}{RGB}{52,152,219}
\definecolor{rowgray}{RGB}{245,245,245}
\definecolor{darkgray}{RGB}{60,60,60}

\hypersetup{
    colorlinks=true,
    linkcolor=headerblue,
    urlcolor=headerblue,
    citecolor=headerblue
}

%%%%%%%%%% STYLE DES CHAPITRES %%%%%%%%%%
\titleformat{\chapter}[hang]{\bfseries\huge\color{headerblue}}{\thechapter.}{0.5em}{}
\titleformat{\section}[hang]{\bfseries\Large\color{headerblue}}{\thesection.}{0.5em}{}
\titleformat{\subsection}[hang]{\bfseries\color{darkgray}}{\thesubsection.}{0.5em}{}

%%%%%%%%%% DÉBUT DU DOCUMENT %%%%%%%%%%
\begin{document}

%%%%%%%%%% PAGE DE GARDE %%%%%%%%%%
\begin{titlepage}
    \centering
    \vspace*{1cm}
    \rule{\textwidth}{1pt}\par
    \vspace{0.4cm}
    {\Large \textsc{I.U.T. A – Département Informatique – Semestre 5}\par}
    \vspace{2cm}
    {\Huge\bfseries S5.B.01 – Déploiement d’une infrastructure réseau\par}
    \vspace{1cm}
    \includegraphics[width=0.3\textwidth]{image/logo-UT-site.png}\par
    \vspace{2cm}
    {\Large Maxence Lagourgue, Celian Allouache, Amine Mahmoudy, Yassine El Mansouri, Thomas Ceolin\par}
    \vfill
    \noindent
    \begin{minipage}[t]{0.5\textwidth}
        \raggedright
        \textbf{Supervision :} M. Kandi
    \end{minipage}%
    \begin{minipage}[t]{0.5\textwidth}
        \raggedleft
        {\large \today}
    \end{minipage}
    \rule{\textwidth}{1pt}
\end{titlepage}

%%%%%%%%%% SOMMAIRE %%%%%%%%%%
\tableofcontents
\clearpage

%%%%%%%%%% PRÉAMBULE %%%%%%%%%%
\chapter*{Préambule}
\addcontentsline{toc}{chapter}{Préambule}

Ce rapport présente le déroulement complet du projet \textbf{S5.B.01 – Déploiement d’une infrastructure réseau}, réalisé dans le cadre du cinquième semestre du BUT Informatique.  
Il retrace, semaine après semaine, les différentes phases d’implémentation, les difficultés rencontrées, ainsi que les solutions techniques mises en œuvre pour assurer la stabilité et la performance du réseau conçu.

L’objectif principal de ce projet est de permettre aux étudiants de mettre en pratique leurs connaissances en administration et configuration d’infrastructures réseau à travers une approche concrète et collaborative.  
Chaque membre du groupe a participé activement à la mise en place du matériel, au câblage, à la configuration des protocoles de routage (OSPF, BGP) et à l’intégration des services essentiels (DHCP, VLAN, adressage IP).

Afin d’assurer une organisation claire et un suivi efficace du travail, nous avons utilisé un tableau \textbf{Trello} tout au long du projet.  
Cet outil de gestion de tâches nous a permis :
\begin{itemize}
    \item de planifier les différentes étapes de réalisation du projet ;
    \item d’attribuer à chaque membre du groupe des tâches précises selon ses compétences ;
    \item de suivre la progression du travail en temps réel ;
    \item et de garantir une répartition équilibrée des responsabilités.
\end{itemize}

Grâce à cette méthode, la coordination entre les membres de l’équipe s’est avérée fluide et productive, favorisant une communication constante et un avancement régulier du projet.

\begin{figure}[H]
    \centering
    \includegraphics[width=0.85\textwidth]{image/Trello.png}
    \caption{Exemple de tableau Trello utilisé pour la répartition et le suivi des tâches}
\end{figure}

Ce préambule introduit également la structure du présent rapport :
\begin{itemize}
    \item Les \textbf{phases d’implémentation}, décrivant les tâches réalisées chaque semaine et les choix techniques retenus.
    \item Les \textbf{analyses de configuration}, expliquant les étapes de déploiement, de test et de correction des erreurs.
    \item Les \textbf{résultats obtenus}, mettant en avant les réussites et les apprentissages techniques issus du projet.
\end{itemize}

Enfin, ce document constitue une trace complète du travail d’équipe réalisé, illustrant notre progression, notre organisation et notre montée en compétence tout au long du projet de déploiement réseau.
\clearpage

%%%%%%%%%% CHAPITRE 6 – PHASE D’IMPLEMENTATION %%%%%%%%%%
\chapter{Phase d’implémentation – Semaine 1 (13 → 19 octobre 2025)}

\section{Objectif de la semaine}
Cette première semaine avait pour objectif de poser les bases de notre projet réseau. Nous devions préparer l’environnement, déployer l’infrastructure initiale et amorcer la configuration du routage dynamique.  

Les principaux objectifs étaient :
\begin{itemize}
    \item Prendre connaissance du plan de travail et inventorier le matériel disponible (routeurs, switches, câbles).  
    \item Réaliser le câblage physique et logique du réseau.  
    \item Définir le plan d’adressage IP pour chaque équipement.  
    \item Initier la configuration du routage dynamique avec le protocole OSPF.  
\end{itemize}

\textbf{Résultat attendu :} un réseau initial stable, câblage organisé, plan d’adressage défini et routage dynamique en cours de mise en place.

\section{Découverte du matériel et contraintes}
Dès le début, nous avons inventorié l’ensemble du matériel afin d’adapter notre architecture aux ressources disponibles.  
Une contrainte majeure est rapidement apparue : \textbf{notre groupe disposait d’un routeur avec une seule interface réseau}, alors que les autres groupes avaient plusieurs interfaces.  

Cette limitation a directement influencé notre topologie, réduisant la possibilité de créer des VLANs multiples sur ce routeur et nécessitant des configurations spécifiques pour maintenir la connectivité entre tous les sous-réseaux.  
Cette situation a renforcé l’importance d’une planification rigoureuse avant toute configuration.

\section{Première tentative de câblage}
Nous avons d’abord réalisé un câblage complet reliant tous les routeurs et switches.  
Cependant, cette première version manquait de clarté et risquait de provoquer des erreurs lors de la manipulation.  

Pour améliorer la lisibilité et la propreté du montage, nous avons décidé de tout recommencer :
\begin{itemize}
    \item Élaboration d’un plan de câblage précis.  
    \item Étiquetage systématique de chaque câble selon sa fonction.  
    \item Réorganisation physique des équipements pour un rendu propre et professionnel.
\end{itemize}

\begin{figure}[H]
    \centering
    \begin{minipage}{0.45\textwidth}
        \centering
        \includegraphics[width=\textwidth]{image/cablagedesordre.png}
        \caption{Premier câblage – désorganisé}
    \end{minipage}\hfill
    \begin{minipage}{0.45\textwidth}
        \centering
        \includegraphics[width=\textwidth]{image/cablageRang.png}
        \caption{Câblage final – organisé et étiqueté}
    \end{minipage}
\end{figure}

\clearpage
\section{Problèmes rencontrés et refonte de la topologie}
Lors de la configuration réseau, plusieurs erreurs ont été identifiées :
\begin{itemize}
    \item Confusion entre les VLAN destinés aux membres et ceux destinés aux adhérents.  
    \item Mauvaise attribution des adresses IP et des passerelles.  
    \item Topologie initiale incorrecte, rendant certains liens inaccessibles.  
\end{itemize}

Ces problèmes étaient principalement liés à une mauvaise anticipation du comportement des routeurs, combinée à la contrainte de l’unique interface sur notre routeur.  
Nous avons donc repensé entièrement la topologie, ajusté les liens WAN et adapté la configuration des VLANs pour correspondre à la réalité du matériel.  

Cette refonte a engendré une perte de \textbf{deux jours de travail}, mais elle a été cruciale pour stabiliser le réseau et mieux comprendre nos erreurs.  
Nous avons ensuite redoublé d’efforts pour rattraper le retard et terminer la configuration des éléments essentiels.

\begin{figure}[H]
    \centering
    \includegraphics[width=0.65\textwidth]{image/shématopologiereseauxadressageip.png}
    \caption{Nouvelle topologie du réseau après refonte et ajustement des VLANs}
\end{figure}

\section{Configuration réseau : IPv4, IPv6, VLAN et passerelles}
Une fois la topologie stabilisée, nous avons repris la configuration complète du réseau :  
\begin{itemize}
    \item Attribution des adresses \textbf{IPv4} sur toutes les interfaces.  
    \item Mise en place des \textbf{VLANs} selon la nouvelle topologie.  
    \item Définition des \textbf{passerelles} pour chaque sous-réseau.  
    \item Début de configuration des adresses \textbf{IPv6} pour assurer la compatibilité dual-stack.  
\end{itemize}

\begin{table}[H]
\centering
\rowcolors{2}{rowgray}{white}
\begin{tabularx}{\textwidth}{@{}lXlX@{}}
\toprule
\rowcolor{headerblue}
\color{white}\textbf{Routeur} & \color{white}\textbf{Interface / VLAN} & \color{white}\textbf{Adresse IPv4 / Masque} & \color{white}\textbf{IPv6 ULA} \\
\midrule
\textbf{R1} & GE8 (WAN) / — & 10.0.20.1 /27 & fd82:f0db:20::2002:a00:1400:1 \\
            & — & 10.0.21.1 /27 & fd82:f0db:21::2002:a00:1500:1 \\
            & GE0 / VLAN2 & 192.168.100.1 /30 & fd82:f0db:2::2002:c0a8:6400:1 \\
\midrule
\textbf{R2} & GE0 / VLAN2 & 192.168.100.2 /30 & fd82:f0db:2::2002:c0a8:6400:2 \\
            & GE4 / VLAN4 & 192.168.110.2 /30 & FD00:110::2/64 \\
            & GE8 (WAN) / — & 192.168.120.2 /30 & FD00:120::2/64 \\
\midrule
\textbf{R3} & GE0 / VLAN2 & 192.168.130.2 /30 & FD00:130::2/64 \\
            & GE4 / VLAN4 & 192.168.110.2 /30 & FD00:110::2/64 \\
\midrule
\textbf{R4} & GE0 / VLAN2 & 192.168.130.1 /30 & FD00:130::1/64 \\
            & GE4 / VLAN4 & 192.168.140.2 /30 & FD00:140::2/64 \\
            & GE8 (WAN) / — & 192.168.120.1 /30 & FD00:120::1/64 \\
\midrule
\textbf{R5} & GE4 / VLAN4 & 192.168.140.1 /30 & FD00:140::1/64 \\
            & GE5 / VLAN5 & 172.0.2.1 /16 & FD00:2::1 \\
            & GE8 (WAN) / — & 10.0.10.1 /27 & fd82:f0db:10::2002:a00:a00:1 \\
            & — / — & 10.0.11.1 /28 & fd82:f0db:11::2002:a00:b00:1 \\
            & — / VLAN2 & 192.168.150.1 /30 & FD00:150::1 \\
\midrule
\textbf{R6} & FE0/1 / — & 172.0.2.2 /16 & FD00:2::2/64 \\
            & FE0/0 / VLAN30 & 10.0.30.1 /28 & fd82:f0db:30::2002:a00:1e00:1 \\
            & FE0/0 / VLAN31 & 10.0.31.1 /28 & fd82:f0db:31::2002:a00:1f00:1 \\
            & — / — & 192.168.150.2 /30 & FD00:150::2 \\
            & — / — & 192.168.150.3 /30 & — \\
            & — / — & 192.168.150.4 /30 & — \\
\bottomrule
\end{tabularx}
\caption{Plan d’adressage IP simplifié avec VLANs et IPv6 pour chaque routeur}
\end{table}

\section{Configuration du protocole OSPF}
Après avoir configuré les adresses IP et les VLAN, nous avons initié le routage dynamique avec le protocole \textbf{OSPF}, garantissant la communication entre tous les sous-réseaux et simplifiant l’extension future du réseau.  

\subsection*{Exemple de configuration sur le Routeur 2}
\begin{enumerate}
    \item \textbf{Préparation du routeur :}
\begin{verbatim}
enable
conf t
hostname R2
\end{verbatim}

    \item \textbf{Création des VLANs et activation IPv6 :}
\begin{verbatim}
vlan database
vlan 2,4
exit
conf t
ipv6 unicast-routing
\end{verbatim}

    \item \textbf{Configuration des interfaces VLAN :}
\begin{verbatim}
interface vlan 2
ip address 192.168.100.2 255.255.255.252
ipv6 address fd82:f0db:2::2002:c0a8:6400:2/64
no shutdown
exit

interface vlan 4
ip address 192.168.110.1 255.255.255.252
ipv6 address FD00:110::1/64
no shutdown
exit
\end{verbatim}

    \item \textbf{Assignation des VLAN aux interfaces physiques :}
\begin{verbatim}
interface GigabitEthernet0
switchport mode access
switchport access vlan 2
exit

interface GigabitEthernet4
switchport mode access
switchport access vlan 4
exit
\end{verbatim}

    \item \textbf{Configuration d’une interface supplémentaire et route par défaut :}
\begin{verbatim}
interface GigabitEthernet8
ip address 192.168.120.1 255.255.255.252
ipv6 address FD00:120::1/64
no shutdown
exit

ip route 0.0.0.0 0.0.0.0 192.168.110.2
\end{verbatim}

    \item \textbf{Activation d’OSPF pour IPv4 et IPv6 :}
\begin{verbatim}
router ospf 2
network 192.168.110.0 0.0.0.255 area 0
network 192.168.120.0 0.0.0.255 area 0
network 192.168.100.0 0.0.0.255 area 1

ipv6 router ospf 3
router-id 2.2.2.2
\end{verbatim}
\end{enumerate}

\section{Bilan de la semaine}
Cette première semaine a été riche en apprentissages et en ajustements :  
\begin{itemize}
    \item \textbf{Points positifs :} organisation du matériel, câblage finalisé, compréhension de la topologie, mise en pratique du protocole OSPF.  
    \item \textbf{Difficultés rencontrées :} contraintes matérielles, topologie initiale incorrecte, VLAN inversés, attribution IP.  
    \item \textbf{Compétences développées :} routage dynamique OSPF, gestion VLAN et sous-réseaux, planification et correction de topologie.
\end{itemize}

---

\chapter{Phase d’implémentation – Semaine 2 (20 → 26 octobre 2025)}

\section{Objectif de la semaine}
Cette semaine a été entièrement consacrée à la **finalisation du routage dynamique, du DHCP IPv4 et du déploiement initial du site web**, tout en corrigeant les problèmes rencontrés dans la topologie et les configurations.  

\textbf{Résultat attendu :} un réseau stable, avec OSPF et BGP fonctionnels, DHCP opérationnel, NAT configuré et site interne amorcé.

Les principaux objectifs étaient :
\begin{itemize}
    \item Finaliser le protocole **OSPF** (IPv4 et IPv6) sur l’ensemble des routeurs.  
    \item Configurer le routage inter-domaine **BGP** entre R5 et R6.  
    \item Déployer et vérifier le service **DHCP IPv4** pour tous les VLAN.  
    \item Configurer le **NAT** sur R5 pour le serveur.  
    \item Commencer le **déploiement du site web** via Docker Compose.  
    \item Corriger les **bugs liés au routage et aux masques** (/28 remplacé par /27 pour tous les VLANs).  
    \item Vérifier la **connectivité complète** entre tous les sous-réseaux.
\end{itemize}

\section{Configuration et finalisation du routage dynamique}
\subsection{OSPF multi-area (IPv4 et IPv6)}
Le réseau a été structuré en architecture **multi-area** pour optimiser la propagation des routes et limiter le trafic sur le backbone central :  
\begin{itemize}
    \item \textbf{Area 0} : backbone central (R1, R2, R3, R4).  
    \item \textbf{Area 1 et Area 2} : zones secondaires correspondant aux VLAN périphériques.  
\end{itemize}

Les ABR ont été identifiés pour relier les zones secondaires à l’Area 0 et chaque routeur annonce ses sous-réseaux dans la zone appropriée.

\subsubsection*{Exemple de configuration sur R3}
\begin{lstlisting}[language=Python]
interface Vlan2
 ip address 192.168.130.1 255.255.255.224  # /27
 ipv6 address FD00:130::1/64
 ipv6 ospf 2 area 0

interface Vlan4
 ip address 192.168.110.2 255.255.255.224  # /27
 ipv6 address FD00:110::2/64
 ipv6 ospf 2 area 0

router ospf 2
 network 192.168.110.0 0.0.0.31 area 0
 network 192.168.130.0 0.0.0.31 area 0
\end{lstlisting}

\subsection{Problèmes rencontrés et solutions}
\begin{itemize}
    \item Chevauchement et insuffisance des adresses pour les VLAN (/28 trop petit → passage à /27).  
    \item Routes manquantes ou mal propagées.  
    \item Bugs sur certaines configurations de routage et NAT côté serveur.
\end{itemize}

\textbf{Solutions apportées :} ajustement des masques à /27 pour tous les VLANs, correction des configurations ABR et des interfaces, vérification du routage côté serveur.

\subsection{Configuration du BGP inter-domaine}
Pour permettre le routage entre deux AS distincts :  
\begin{itemize}
    \item R5 : AS 5  
    \item R6 : AS 6  
\end{itemize}

\subsubsection*{Configuration sur R5}
\begin{lstlisting}[language=Python]
router bgp 5
 bgp log-neighbor-changes
 network 172.0.0.0
 neighbor 172.0.2.2 remote-as 6
\end{lstlisting}

Les tests via \texttt{show ip bgp summary} et \texttt{show ip route} ont confirmé la propagation correcte des préfixes entre R5 et R6.

\section{Déploiement et configuration des services essentiels}
\subsection{DHCP IPv4}
Le serveur DHCP a été configuré pour chaque VLAN avec :
\begin{itemize}
    \item Création des pools d’adresses IPv4 (/27).  
    \item Définition des options DHCP : passerelle et DNS interne.  
    \item Vérification sur plusieurs clients pour confirmer l’attribution correcte des adresses.
\end{itemize}

\subsection{NAT sur R5}
Le NAT a été configuré pour permettre l’accès externe au serveur :
\begin{itemize}
    \item Traduction d’adresses privées vers une adresse publique.  
    \item Tests de connectivité et vérification du bon fonctionnement pour le serveur.
\end{itemize}

\subsection{Routage et corrections côté serveur}
Le serveur a été configuré avec des routes statiques et dynamiques pour permettre :
\begin{itemize}
    \item La communication avec tous les VLANs.  
    \item La compatibilité avec le NAT et BGP.  
    \item La correction des bugs détectés lors des premiers tests.
\end{itemize}

\subsection{Début du déploiement du site web}
Pour amorcer la mise en ligne du site interne :
\begin{itemize}
    \item Création des containers via \textbf{Docker Compose}.  
    \item Test de l’accès réseau aux containers depuis les VLANs.  
    \item Projet \textbf{Nailloux manquant}, bloquant la finalisation complète du déploiement.
\end{itemize}

\subsection{Tests de connectivité}
\begin{itemize}
    \item \texttt{ping} entre tous les routeurs et clients.  
    \item Vérification des tables de routage OSPF et BGP.  
    \item Test de l’attribution DHCP et du NAT pour le serveur.  
    \item Validation que les VLANs communiquent correctement entre eux.
\end{itemize}

\section{Bilan de la semaine}
Cette semaine a permis de consolider le réseau et d’avancer sur le déploiement du site :  
\begin{itemize}
    \item OSPF multi-area fonctionnel en IPv4 et IPv6.  
    \item BGP inter-domaine opérationnel entre R5 et R6.  
    \item DHCP IPv4 entièrement déployé et fonctionnel.  
    \item NAT sur R5 permettant l’accès au serveur.  
    \item Corrections des bugs de routage et ajustement des masques (/27).  
    \item Début du déploiement du site web via Docker Compose.  
    \item Projet Nailloux manquant pour la finalisation complète.
\end{itemize}

Le réseau est désormais \textbf{fonctionnel, stable et prêt pour la suite du projet}, avec routage dynamique, attribution d’adresses DHCP, NAT opérationnel et démarrage du site interne.



\end{document}
